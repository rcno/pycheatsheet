
%%%%%%%%%%%%%%%%%%%%%%%%%%%%%%%%%%%%%%%%%%%%%%%%%%%%%%%%

%% Author: Ragnhild C. Noven

%% Inspired by https://tex.stackexchange.com/questions/8827/preparing-cheat-sheets
%%%%%%%%%%%%%%%%%%%%%%%%%%%%%%%%%%%%%%%%%%%%%%%%%%%%%%%%%%%%%%%%%

\documentclass[11pt]{article}
\usepackage{multirow,multicol,makecell}

\usepackage[a4paper,margin=0.55in,landscape]{geometry}
\usepackage{amsmath,amsthm,amsfonts,amssymb}
\usepackage{color,graphicx,overpic}
\usepackage{hyperref}
\usepackage{enumitem,kantlipsum}

%% NB: important that this comes first, otherwise get option clash
\usepackage[table]{xcolor}

% Turn off header and footer
\pagestyle{empty}

% Redefine section commands to use less space
\makeatletter
%\renewcommand{\section}{\@startsection{section}{1}{0mm}%
%                                {-1ex plus -.5ex minus -.2ex}%
%                                {0.5ex plus .2ex}%
%                                {\normalfont\large\bfseries}}
\renewcommand{\subsection}{\@startsection{subsection}{2}{0mm}%
                                {-1explus -.5ex minus -.2ex}%
                                {0.5ex plus .2ex}%
                                {\normalfont\normalsize\bfseries}}
\renewcommand{\subsubsection}{\@startsection{subsubsection}{3}{0mm}%
                                {-1ex plus -.5ex minus -.2ex}%
                                {1ex plus .2ex}%
                                {\normalfont\small\bfseries}}
\makeatother

% Define BibTeX command
\def\BibTeX{{\rm B\kern-.05em{\sc i\kern-.025em b}\kern-.08em
    T\kern-.1667em\lower.7ex\hbox{E}\kern-.125emX}}

% Don't print section numbers
\setcounter{secnumdepth}{0}

% For custom table columns
\usepackage{array}

\setlength{\parindent}{0pt}
\setlength{\parskip}{0pt plus 0.5ex}

%My Environments
\newtheorem{example}[section]{Example}

%-----------------------------------------------------------------------

\usepackage{listings,bookmark}


\definecolor{codegreen}{rgb}{0,0.6,0}
\definecolor{codegray}{rgb}{0.5,0.5,0.5}
\definecolor{codepurple}{rgb}{0.58,0,0.82}
\definecolor{backcolour}{rgb}{0.95,0.95,0.92}
 
\lstdefinestyle{mystyle}{
    backgroundcolor=\color{backcolour},   
    commentstyle=\color{codegreen},
    keywordstyle=\color{blue},
    numberstyle=\tiny\color{codegray},
    stringstyle=\color{codepurple},
    basicstyle=\ttfamily\footnotesize,
    breakatwhitespace=false,         
    breaklines=true,                 
    captionpos=b,
    frame=single,
    keepspaces=true,                 
    numbers=none,                    
    numbersep=5pt,                  
    showspaces=false,                
    showstringspaces=false,
    showtabs=false,                  
    tabsize=2,
    columns=fullflexible
  }
 
\lstset{style=mystyle}


%%%%%%%%%%%%% Make sections have coloured background %%%%%%%%%%%%%%%

\usepackage[T1]{fontenc}
\usepackage[utf8]{inputenc}
\usepackage{tikz}
\usepackage{titlesec}


\titleformat{\section}{\large \bfseries}{}{2em}{\colorsection}
\titleformat{name=\section,numberless}{\large \bfseries}{}{0em}{\colorsectionnonumber}

%\titlespacing{<command>}{<left>}{<before-sep>}{<after-sep>}
\titlespacing{\section}{-4px}{0.5em}{1em}

% NB: crucial to use \columnwidth and not \textwidth
\newcommand{\colorsection}[1]{%
\colorbox{lightgray}{\parbox{\dimexpr\columnwidth-2\fboxsep-2\fboxrule\relax}{\thesection\ #1}}}

\newcommand{\colorsectionnonumber}[1]{%
\colorbox{lightgray}{\parbox{\dimexpr\columnwidth-2\fboxsep-2\fboxrule\relax}{#1}}}


% NB: raggedright is important here
\titleformat{\subsection}{\large \raggedright \bfseries}{}{0em}{}

\setlength{\columnsep}{18pt}
%------------------------------------------------------------------------------------------------------------------------


\begin{document}

\centering

{\scshape\huge \bfseries Python for data science} \\[4pt]
{\large Ragnhild C.~Noven}

\vspace{6pt}

\raggedright
\footnotesize
\begin{multicols*}{3}


% multicol parameters
% These lengths are set only within the two main columns
%\setlength{\columnseprule}{0.25pt}
\setlength{\premulticols}{1pt}
\setlength{\postmulticols}{1pt}
\setlength{\multicolsep}{30pt}
\setlength{\columnsep}{2pt}

\section{Basic imports}
\vspace{2pt}
\begin{lstlisting}[language=Python,linewidth=0.95\linewidth]
import numpy as np 
import scipy as sp 
\end{lstlisting}

\section{Navigation}
\vspace{2pt}
\begin{minipage}{0.95\linewidth}
  \centering
\begin{lstlisting}[language=Python]
import os     
os.getcwd()
\end{lstlisting}
\end{minipage}


\section{Input and output}

\begin{itemize}[wide = 4pt, labelindent=\parindent]
  
\item \textbf{Read} lines from file
\vspace{2pt}
\begin{lstlisting}[language=Python,linewidth=0.95\linewidth]
import sys
with open(sys.argv[1]) as f:
    flines = f.readlines() 
for line in flines: 
    print line 
\end{lstlisting}

\item \textbf{Write} to file
\vspace{2pt}
\begin{lstlisting}[language=Python,linewidth=0.95\linewidth]
with open('filename.txt','w') as f:
    f.write(mystr)
\end{lstlisting}
\end{itemize}



\section{Lists}
\begin{itemize}[wide = 4pt, labelindent=\parindent]
\item \textbf{Merge} a list into another using \verb+mylist.extend(taillist)+ to 
\end{itemize}


\section{Dictionaries}
\begin{itemize}[wide = 4pt, labelindent=\parindent]

\item Iterate over key,value pairs in dictionary
  \vspace{2pt}
\begin{lstlisting}[language=Python,linewidth=0.95\linewidth]
[func(key,value) for key,value in dict.iteritems()]
[func(key,value) for key,value in zip(dict.keys(),dict.values())]
\end{lstlisting}

  \item Get the value at key, and if key is not in dictionary, insert it with a default value:
\vspace{2pt}
\begin{lstlisting}[language=Python,linewidth=0.95\linewidth]
mydict.setdefault(key,defval)
\end{lstlisting}
\end{itemize}

\section{Strings}
\begin{itemize}[wide = 4pt, labelindent=\parindent]
\item Basic functions:
\begin{lstlisting}[language=Python,linewidth=0.95\linewidth]
>>> "%s is %.2f" % ("Foo",3.567)
'Foo is 3.57'
str.count(substr)
separator.join(iterable)
>>> 'Py' in 'Python'
True
# returns -1 if not found
str.find(substr)
str.replace(old,new)
str.split(sepstr)
str.strip(schars)

str.startswith("substr")
\end{lstlisting}
  
\end{itemize}

\section{Unicode}
\begin{itemize}[wide = 4pt, labelindent=\parindent]
\item Unicode looks like \verb+\u2119+, has integers that uniquely represent symbols.
\item The conversion from unicode to bytes can be done by different \textit{encodings}, such as 'ascii', 'utf8', 'windows-1252' etc. 
\item \verb+unicode.encode -> bytes+
\item \verb+bytes.decode -> unicode+
\item If you have a unicode string and get errors, try using \verb+'xmlcharrefreplace'+ as the second argument to \verb+encode+. 
\end{itemize}

See \url{https://nedbatchelder.com/text/unipain.html}

\section{Objects}
\begin{itemize}
\item \textbf{Introspection}: use \verb+dir()+ or \verb+type()+.
\end{itemize}

\section{Don't ask permission}
\begin{lstlisting}[language=Python,linewidth=0.95\linewidth]
try:
    do_something(x)
except Exception, e:
    print e
\end{lstlisting}

\section{Operators}
% \begin{lstlisting}[language=Python,linewidth=0.95\linewidth]
    \rowcolors{2}{gray!25}{white}
\begin{tabular}{>{\ttfamily \large}l>{\ttfamily}l}
  \hline
  \multicolumn{1}{l}{Operator} & \multicolumn{1}{l}{Meaning} \\
\hline
& and, or, not, in \\
** &  Exponent \\
\%  &  Modulus  \\
!= &  Not equal  \\
\&  &  Bitwise AND  \\	
\verb+|+  &  Bitwise OR  \\	
\verb+~+  &  Bitwise NOT  \\	
\verb+^+  &  Bitwise XOR  \\
\hline
\end{tabular}  
%\end{lstlisting}
  
\section{Graphs}

\subsection{Basic graphs}
Start with importing the pyplot package
\vspace{2pt}
\begin{lstlisting}[language=Python,linewidth=0.95\linewidth]
import matplotlib.pyplot as plt
\end{lstlisting}
Then you can use
\vspace{2pt}
\begin{lstlisting}[language=Python,linewidth=0.95\linewidth]
##create empty plot 
plt.subplot(1,1,1)
plt.plot([1,2,3])
plt.show()
##NB: clear plot 
plt.close()
\end{lstlisting}
to plot lists.

\subsection{Barplot}
\vspace{2pt}
\begin{lstlisting}[language=Python,linewidth=0.95\linewidth]
plot = plt.subplot()
b2 = plot.bar([1,2,3],[4,5,6],color='g',width=0.5)
b1 = plot.bar([0.5,1.5,2.5],[1,2,3],width=0.5)
plt.show()
\end{lstlisting}

\subsection{Histogram}
Let \verb+mydict+ be a dictionary of counts. 
\vspace{2pt}
\begin{lstlisting}[language=Python,linewidth=0.95\linewidth]
pos = np.arange(len(mydict))
plt.bar(pos, [x[1] for x in mydict.iteritems()])
plt.xticks(pos, [x[0] for x in mydict.iteritems()])
plt.show()
\end{lstlisting}

\subsection{Multiple plots}
\begin{lstlisting}[language=Python,linewidth=0.95\linewidth]
fig = plt.figure()

f1 = fig.add_subplot(221)
data.hist(bins=25,color='navy',ax=f1)
f1.set_title('Histogram')
f1.set_xlabel('Xlabel')

f2 = fig.add_subplot(222)
f3 = fig.add_subplot(223)
f4 = fig.add_subplot(224)
\end{lstlisting}


\section{Modules}

\begin{itemize}[wide = 4pt, labelindent=\parindent]
\item \textbf{Each module has its own namespace}. So if we do \verb+import mymod+, then we can use items in mymod via \verb+mymod.modfunc()+. It is possible to import all items from a module directly via \verb+from mymod import *+, but this can lead to name mangling, and should be used with care.   
    
\item Any file ``myfile.py'' is a \textbf{module}, and can be imported by using \verb+import myfile+.

\item Can add code that is only called when the module is run as a main \textbf{script} (as \verb~python module.py~):
  \vspace{2pt}
\begin{lstlisting}[linewidth=0.95\linewidth]
if __name__ == "__main__":
    some_code_here
\end{lstlisting}

\item Import a module from an \textbf{absolute path}:
\begin{lstlisting}[language=Python,linewidth=0.95\linewidth]
import sys
import os
sys.path.append(os.path.expanduser('~') + "/path/to/file")
\end{lstlisting}

\item Import a module from a \textbf{relative path}:
\begin{lstlisting}[language=Python,linewidth=0.95\linewidth]
  from ..pkg import module
\end{lstlisting}
imports module.py in the folder pkg, where pkg is in the parent directory of the current location. 

\item \textbf{NB:} if the \verb~__init__.py~ file is not present, then can't import module files from other folders, solution is to add this file (it may be left empty).

  \section{Packages}
  
\item A \textbf{package} is a directory with a collection of modules (files), plus an  \verb+__init__.py+ file. Packages can be used in two ways, either write
  \vspace{2pt}
\begin{lstlisting}[language=Python,linewidth=0.95\linewidth]
import mypackage
mypackage.module.function()
\end{lstlisting}
where module corresponds to a file module.py, or use
\begin{lstlisting}[language=Python,linewidth=0.95\linewidth]
from mypackage import module
module.function()
\end{lstlisting}

\end{itemize}

\section{Pandas}
\begin{itemize}[wide = 4pt, labelindent=\parindent]
  \item Basic functions for data frames
\begin{lstlisting}[language=Python,linewidth=0.95\linewidth]
import pandas as pd
df=pd.read_csv("myfile.csv", header=None, sep=";",names=["first",'c2']))
# toy dataframe
df = pd.DataFrame({"c1": [1,2,3,4], "c2": [1,1,2,2]})
df.shape
df.columns
df.dtypes
df.describe()
df.head(10)
\end{lstlisting}

 \item Filtering data frames

\begin{lstlisting}[language=Python,linewidth=0.95\linewidth]    
# indexing
df.ix[0,1]
# extract columns
df[["c1"]]
df.filter(items=["c1","c2"])
# extract rows 
df.filter(items = [2,3],axis=0)
# select elements
df[df.c1 < 5]
df.query("c1 > 4")\end{lstlisting}

So to get the fourth row, can do either \verb+df.ix[3,:]+ or \verb+ users.filter(items=[3],axis=0)+. 

\item For data series, can do
\begin{lstlisting}[language=Python,linewidth=0.95\linewidth]
series[series>3].index
series.sort_values(ascending=True)
\end{lstlisting}
  
\item Missing data: \verb+df[df.c1.isnull()]+, can do \verb+df = df[df.c1.notnull()]+.  

\item Data summaries
\begin{lstlisting}[language=Python,linewidth=0.95\linewidth]
df.c1.hist(bins=20)
plt.show()  
\end{lstlisting}

\section{Split, apply, combine in Pandas}

\end{itemize}

\section{Grammar of data}
    \rowcolors{2}{gray!25}{white}
\vspace{1pt}
\begin{tabular}{>{\ttfamily}l>{\ttfamily}l>{\ttfamily}l>{\ttfamily}l}
  \hline
  \multicolumn{1}{l}{\textbf{Verb}} &  \multicolumn{1}{l}{\textbf{Pandas}} & \multicolumn{1}{l}{ \textbf{SQL}} \\
  \hline
  Query & query() & SELECT WHERE \\
  Sort &  sort() & ORDER BY \\
  Projection & [[]] & SELECT COLUMN \\
  \multirow{2}{*}{\begin{minipage}{0.2\columnwidth}{Select-\\ distinct } \end{minipage} } & unique() & \multirow{2}{*}{\begin{minipage}{0.3\columnwidth}{SELECT DISTINCT  \\ COLUMN} \end{minipage} }\\[10pt]
  Assign & assign & ALTER/UPDATE \\
  Aggregate &\multirow{2}{*}{\begin{minipage}{0.3\columnwidth}{describe(),  \\ mean(), max() } \end{minipage} } & \multirow{2}{*}{\begin{minipage}{0.3\columnwidth}{COUNT(),AVG(),\\MAX(), SUM() } \end{minipage} } \\[10pt]
  Sample & sample() & RAND() \\
  Group-agg & agg,count,mean & GROUP BY \\
  DELETE & drop & DELETE/WHERE
\end{tabular}

\vspace{5pt}

\section{Numpy arrays }

\begin{itemize}[wide = 4pt, labelindent=\parindent]
\item Create an array using \verb+array([])+. You always need a list directly within \verb+array()+. For example \verb+array([[1,2],[3,4]])+ creates a matrix with rows equal to the inner lists. 
\item Create an array of zeros with a given size: \verb+zeros((n1,n2))+. 
\item Array with elements of different lengths, do \verb+array([[0,1],[1]],object)+, which creates a different type of array. 
\item Operations: 
  \begin{itemize}
  \item Use \verb+dot(a,b)+ for any type of matrix multiplication, note that vectors do not distinguish between row/column versions. 
  \item Use \verb+add(a,b)+ for adding arrays \textit{elementwise} and \verb+subtract(a,b)+ for subtracting elementwise. 
  \item Note that these operations will work on lists, but they will output an array, so try to work consistently with arrays within a function. 
  \end{itemize}
\item Converting from arrays:
  \begin{itemize}
  \item Use \verb+list()+ to convert an array to a list. A matrix will become a list of arrays containing the rows. 
  \item Use \verb+float()+ to convert an array of length 1 to a floating-point number.  
  \end{itemize}
\item Pitfalls:
  \begin{itemize}
  \item Printing an array just returns the vector/matrix/\ldots as a list, so something could be an array but look like a list. 
  \end{itemize}
\end{itemize}

\section{Pickle}
Saving objects, useful for dataframes etc.
\begin{lstlisting}[language=Python,linewidth=0.95\linewidth]
# Save object
import cPickle
with open('pfile.pkl','wb') as f:
    cPickle.dump(object,f)

# Read saved object
with open('pfile.pkl','r') as f:
    object = cPickle.load(f)
\end{lstlisting}

\section{Timing}
\begin{lstlisting}[language=Python,linewidth=0.95\linewidth]
import time
start = time.time()
print("hello")
end = time.time()
print(end - start)
\end{lstlisting}



\section{Regexp in Python}

\begin{lstlisting}[language=Python,linewidth=0.95\linewidth]
## replace parts of a string
nstr = re.sub(pattern, replacement, str, max=0)
\end{lstlisting}

\rowcolors{2}{gray!25}{white}
\begin{tabular}{>{\ttfamily \large}l>{\ttfamily}l}
  \hline
  \multicolumn{1}{l}{Pattern} & \multicolumn{1}{l}{Matches} \\
\hline
\verb+^+	      & Beginning of line \\
\verb+$+	& End of line \\
.	& Exactly one character \\
\verb+[...]+	& Any one character in the brackets \\
\verb+[^...]+	& Any one character \textit{not} in brackets\\
x*	& 0 or more times x\\
x+	& 1 or more times x\\
x?	& 0 or 1 times x\\
\verb+\w+	& Word characters\\
\verb+\W+	& Nonword characters\\
\verb+\s+	& Whitespace  \\
\verb+\d+	& Digits. Equivalent to [0-9] \\
\end{tabular}    


\end{multicols*}

\end{document}